\documentclass{report}

%pacchetti
\usepackage[utf8]{inputenc}
\usepackage[italian]{babel}
%permette ai riferimenti di essere cliccabili
\usepackage{hyperref}
%permette di fare riferimento a un elemento tramite il suo nome
\usepackage{nameref}
%impostazione dei margini del documento
\usepackage[margin=1in]{geometry}
%inserimento di immagini
\usepackage{graphics}

%inserimento di pezzi di codice
\usepackage{listings}
%colori per il codice
\usepackage{xcolor}

%impostazioni dei collegamenti ipertestuali
\hypersetup{
    colorlinks=true,
    linkcolor=blue,
    filecolor=blue,
}

\urlstyle{same}

%impostazione della profondità a cui mostrare le parti del documento nell'indice
\setcounter{tocdepth}{3}
\setcounter{secnumdepth}{3}

%setup colori e visualizzazione codice
\definecolor{codegray}{rgb}{0.5,0.5,0.5}
\definecolor{lgray}{rgb}{0.95,0.95,0.95}

\lstdefinestyle{mystyle}{
    backgroundcolor=\color{lgray},   
    commentstyle=\color{cyan},
    keywordstyle=\color{blue},
    numberstyle=\color{codegray},
    stringstyle=\color{orange},
    breakatwhitespace=false,         
    breaklines=true,                                   
    keepspaces=true,                 
    numbers=left,                    
    numbersep=5pt,                  
    showspaces=false,                
    showstringspaces=false,
    showtabs=false,                  
    tabsize=2
}

\lstset{style=mystyle}

%disabilitazione dell'indentazione a inizio paragrafo
\setlength{\parindent}{0pt}

\title{Meteo}
\author{Matilde Labruzzo 987069}
\date{PWM - secondo semestre A.A. 2021/2022}

\begin{document}

\maketitle

% Indice
\tableofcontents

\chapter{Introduzione}

\section{Analisi dei requisiti}

\subsection{Destinatari}

Grazie a una interfaccia semplice ed intuitiva, qualsiasi tipo di utente può accedere e utilizzare facilmente l'applicazione, senza alcun tipo di livello di esperienza richiesto.
Infatti, tramite la barra di ricerca, l'utente può cercare e trovare le informazioni metereologiche riguardanti la città desiderata.
Inoltre, sempre attraverso la barra di ricerca, nella sezione autenticazione, può accedere a maggiori servizi, come ad esempio personalizzare le città preferite.

\vspace{5mm}

Nella versione attuale, non sono posti particolari vincoli di banda grazie alla sola presenza degli elementi utili alla
navigazione, riducendo così la latenza il più possibile. Inoltre, le immagini relative alle condizioni metereologiche delle città
visualizzate saranno caricate in modalità asincrona, riducendo al minimo il periodo in cui la pagina non risponde agli input
dell'utente.\\
Si consiglia la navigazione via PC per una maggiore semplicità di lettura, ma è comunque possibile utilizzare un
telefono grazie alla responsività degli elementi presenti nelle pagine.

\vspace{5mm}

Gli utenti che accedono alla piattaforma web sono spinti da motivazioni prettamente personali, in particolare dalla necessità di cercare informazioni metereologiche riguardo a una o più città.
Di conseguenza, l'applicazione è progettata in modo tale da fornire le informazioni su richiesta esplicita dell'utente, attraverso la barra di navigazione o l'area personale, previa registrazione e login, in cui viene data la
possibilità di salvare le città preferite.

\subsection{Modello di valore}

L'applicazione si contraddistingue per essere intuitiva e veloce da usare, elementi sempre graditi durante l'esperienza utente.

\vspace{5mm}

Grazie alla disponibilità in tempo reale di informazioni relative al tempo, quest'applicazione potrebbe essere integrata con
un'API in grado di fornire informazioni a sistemi automatici usati per svolgere operazioni in base al tempo atmosferico,
risparmiando così ingenti somme di denaro in sensori e cablaggi.\\
Benché attualmente non presenti, sarebbe facile inserire banner pubblicitari o sponsor di sorta grazie alla struttura modulare
del progetto.\\
Entrambi questi elementi accrescerebbero notevolmente il valore economico dell'applicazione in dipendenza, rispettivamente, al
numero di sistemi automatici collegabili per utente o al numero di banner inseriti, motivo per cui risulta difficile formulare
una stima esatta di valore economico.

\subsection{Flusso dei dati}

Il flusso dei dati all'interno dell'applicazione é unicamente in formato JSON: lo scambio di dati tra client e server, tra
client e API o tra server e API é infatti interamente gestito attraverso l'inoltro di oggetti e stringhe JSON.\\
Grazie ad una struttura RESTful, le comunicazioni saranno unicamente aperte dal client nel momento in cui necessiterà di una risorsa
(come il meteo relativo ad una città oppure un'altra pagina del sito), il quale richiederà ciò di cui necessita tramite
oggetti JSON inviati al server, ricevendone altri in risposta che permetteranno di agiornare un frammento della pagina o di effettuare
il reindirizzamento.

\vspace{5mm}

I contenuti salvati sono interamente archiviati lato server tramite l'uso di un database MongoDB ad eccezione della preferenza
espressa dall'utente per cambiare la pagina in \emph{dark mode} oppure in \emph{light mode}, la quale è salvata localmente attraverso
localStorage.

\vspace{5mm}

Allo stato attuale, il progetto prevede solo costi per la manutenzione in up del server, senza che siano necessari particolari
interventi di manutenzione periodici.

\vspace{5mm}

Il progetto utilizza unicamente librerie e API reperibili gratuitamente, ma sarebbe perfettamente possibile modificarle ed
adottarne di closed source a pagamento con pochi e semplici aggiustamenti grazie ad una gestione modulare del codice.

\subsection{Aspetti tecnologici}

La trasmissione di dati può essere effettuata in chiaro ad eccezione della password usata per accedere alla propria area
personale, la quale viene inviata sotto forma di SHA256 per evitare che sia leggibile.

\vspace{5mm}

Il database prevede la creazione di una singola collezione in cui ogni documento contiene:
\begin{itemize}
    \item \emph{\_id}: id univoco assegnato dal database in automatico;
    \item \emph{user}: username;
    \item \emph{email}: email dell'utente;
    \item \emph{pwd}: password dell'utente sotto forma di digest SHA256;
    \item \emph{pref}: array contenente le città messe tra i preferiti dall'utente; può essere vuoto.
\end{itemize}

\vspace{5mm}

Tecnologie utilizzate:
\begin{itemize}
    \item \emph{HTML 5}: realizzazione della struttura delle pagine;
    \item \emph{Bootstrap e CSS3}: gestione degli stili;
    \item \emph{Swiper API}: realizzazione degli swiper presenti all'interno delle pagine;
    \item \emph{JavaScript}: realizzazione delle richiese al server e alle API, nonché del toggle della darkmode e l'aggiornamento
          di frammenti di pagina con dati ricevuti dal server;
    \item \emph{Node JS}: implementazione del server con tutte le sue funzionalità (caricamento di tutte le pagine,
          invio di oggetti JSON per aggiornare frammenti di pagina, interrogazione dell'API per ottenere il nome delle città e delle
          aree geografiche);
    \item \emph{Express}: framework utilizzato per semplificare il deploy del server;
    \item \emph{JSON}: formato usato per la trasmissione dei dati;
    \item \emph{localStorage}: usato per il salvataggio della preferenza utente per la visualizzazione della pagina (dark o light mode);
    \item \emph{MongoDB}: database utilizzato per salvare le informazioni relative agli utenti registrati sull'app;
    \item \emph{API}: nel progetto sono state utilizzate 4 API:
          \begin{itemize}
              \item \emph{openWeather}: API usata per ottenere le condizioni meteo in tempo reale;
              \item \emph{pexels}: API utilizzata per reperire le immagini mostrate nelle pagine;
              \item \emph{spott}: API utilizzata per ottenere i nomi delle città mondiali e per permettere la ricerca incrementale;
              \item \emph{rest countries}: API utilizzata per ottenere i nomi delle capitali delle diverse aree geografiche.
          \end{itemize}
\end{itemize}
\chapter{Interfacce}

\section{Index.ejs}
\chapter{Architettura}
\chapter{Codice}
\chapter{Possibili sviluppi e conclusioni}

L'applicazione si occupa di fornire le condizioni meteo in tempo reale relative ad una o più città in tutto il mondo.

\vspace{5mm}

Alcuni possibili spunti di sviluppo per accrescere l'efficienza, la sicurezza e l'estetica dell'applicazione potrebbero essere:
\begin{itemize}
    \item introduzione di librerie più efficienti rispetto a quelle attualmente integrate;
    \item efficentamento del codice JavaScript attualmente integrato attraverso un maggiore studio delle funzioni offerte ed eventuale
          introduzione di apposite librerie per svolgere tali scopi;
    \item introduzione di web worker per effettuare le richieste asincrone al server tramite fetch, caricare le immagini e effettuare
          il calcolo dello SHA256 delle password inserite dall'utente;
    \item introduzione di un sistema di comunicazione con chiavi SSL per l'utilizzo del protocollo HTTPS;
    \item introduzione di una navbar apposita per i dispositivi mobili.
\end{itemize}

\vspace{5mm}

Alcuni spunti di sviluppo per accrescere il valore commerciale dell'applicazione sono riassumibili in:
\begin{itemize}
    \item introduzione di pubblicità all'interno del sito;
    \item introduzione di un'API per fornire ad altri sistemi il meteo in tempo reale.
\end{itemize}
\chapter{Bibliografia}

\begin{itemize}
    \item API pexels: \url{https://www.pexels.com};
    \item API openWeather: \url{https://www.pexels.com};
    \item API città: \url{https://www.spott.dev/};
    \item API paesi e capitali: \url{https://restcountries.com/};
    \item Libreria usata per generare digest SHA256: \url{https://cdnjs.cloudflare.com/ajax/libs/crypto-js/3.1.2/rollups/sha256.js};
    \item Repository github con l'intero progetto e la documentazione con LaTex sorgente: \url{};
    \item Bootstrap: \url{https://getbootstrap.com};
    \item Swiper: \url{https://swiperjs.com/swiper-api};
\end{itemize}

\end{document}