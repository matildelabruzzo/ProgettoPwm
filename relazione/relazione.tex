\documentclass{report}

%pacchetti
\usepackage[utf8]{inputenc}
\usepackage[italian]{babel}
%permette ai riferimenti di essere cliccabili
\usepackage{hyperref}
%permette di fare riferimento a un elemento tramite il suo nome
\usepackage{nameref}
%impostazione dei margini del documento
\usepackage[margin=1in]{geometry}
%inserimento di immagini
\usepackage{graphics}

%inserimento di pezzi di codice
\usepackage{listings}
%colori per il codice
\usepackage{xcolor}

%impostazioni dei collegamenti ipertestuali
\hypersetup{
    colorlinks=true,
    linkcolor=blue,
    filecolor=blue,
}

\urlstyle{same}

%impostazione della profondità a cui mostrare le parti del documento nell'indice
\setcounter{tocdepth}{3}
\setcounter{secnumdepth}{3}

%setup colori e visualizzazione codice
\definecolor{codegray}{rgb}{0.5,0.5,0.5}
\definecolor{lgray}{rgb}{0.95,0.95,0.95}

\lstdefinestyle{mystyle}{
    backgroundcolor=\color{lgray},   
    commentstyle=\color{cyan},
    keywordstyle=\color{blue},
    numberstyle=\color{codegray},
    stringstyle=\color{orange},
    breakatwhitespace=false,         
    breaklines=true,                                   
    keepspaces=true,                 
    numbers=left,                    
    numbersep=5pt,                  
    showspaces=false,                
    showstringspaces=false,
    showtabs=false,                  
    tabsize=2
}

\lstset{style=mystyle}

%disabilitazione dell'indentazione a inizio paragrafo
\setlength{\parindent}{0pt}

\title{Meteo}
\author{Matilde Labruzzo 987069}
\date{Corso di Programmazione Web e Mobile - A.A. 2021/2022}

\begin{document}

\maketitle

% Indice
\tableofcontents

\chapter{Introduzione}

\section{Analisi dei requisiti}

\subsection{Destinatari}

\subsection{Modello di valore}

\subsection{Flusso dei dati}

\subsection{Aspetti tecnologici}
\chapter{Interfacce}

Per realizzare le pagine sono stati usati \emph{HTML 5} e \emph{bootstrap}.\\
Gli elementi di personalizzazione, come il nome utente nella navbar una volta autenticati o il meteo
in tempo reale delle città salvate tra i preferiti, sono inseriti utilizzando \emph{EJS}.

\section{Index.ejs}

Nella parte superiore dell'index come prima cosa troviamo una navbar semplice con elementi completamente intuitivi: a partire da sinistra si ha il nome e il logo dell'applicazione,
e il tasto home; grazie al menù a tendina \emph{Aree} è possibile navigare per aree geografiche (i continenti) e cercare informazioni sulle città relative ad esse; cliccando invece sul menù {Autenticazione}
l'utente può accedervi o registrarsi; a seguito del login, questo menù viene sostituito con un altro per poter effettuare il logout e per poter accedere all'area personale, dove sono salvate le città
preferite dall'utente e dove quest'ultimo può cambiare le informazioni di accesso. \\
Sempre sulla navbar troviamo un toggle che dà la possibilità all'utente di cambiare la modalità di visualizzazione,
\emph{light mode} o \emph{dark mode} a seconda della sua preferenza. Sulla destra troviamo invece una barra di ricerca.

\vspace{5mm}

Al di sotto della navbar troviamo
\chapter{Architettura}

\section{Struttura del sito}

\begin{figure}[ht]
    \centering
    \resizebox{\textwidth}{!}{\includegraphics{img/architettura.png}}
    \caption{Architettura gerarchica delle pagine}
\end{figure}

Il punto di accesso al sito è pensato per essere costituito dall'index. Attraverso di essa sarà infatti
possibile accedere tramite uno o più passaggi a tutte le pagine del sito (la navigazione da una pagina all'altra rimane comunque
sempre disponibile attraverso la navbar, a prescindere da dove ci si trovi nel sito).

\vspace{5mm}

Per accedere alla pagina \emph{AreaPersonale.ejs} e a qualunque funzione che richieda di essere loggati,
sarà necessario aver effettuato realmente il login: in caso contrario, la richiesta verrà rifiutata dal
server.

\section{Routing lato server}

Le route previse dal server sono:
\begin{itemize}
    \item \emph{/} (GET): index. Nel caso in cui si sia autenticati (attraverso un cookie), verrà modificata la
          navbar in modo da mostrare un menù a tendina per entrare nell'area personale al posto di quello per l'autenticazione;
          questo viene realizzato modificando il JSON inviato a EJS per renderizzare la pagina;
    \item \emph{/login} (GET): pagina di login. Nel caso il login vada a buon fine, si verrà reindirizzati all'index; altrimenti,
          si verrà reindirizzati a \emph{/login?auth=fail}, query che permette di mostrare un messaggio di errore;
    \item \emph{/registrazione} (GET): pagina per la registrazione. Sia che la registrazione vada a buon fine sia che dia errore,
          verrà mostrato un messaggio in relazione al risultato ottenuto dal server. Per effettuare il login bisognerà usare la pagina
          apposita;
    \item \emph{/verificaCredenziali} (GET): indirizzo raggiunto dal form nella pagina di login per verificare le credenziali
          inserite;
    \item \emph{/creaUtente} (POST): indirizzo raggiunto per creare un nuovo utente;
    \item \emph{/areaPersonale} (GET): pagina relativa all'area personale;
    \item \emph{/aggiornaDati} (POST): indirizzo raggiunto per modificare i dati relativi all'utente registrato;
    \item \emph{/aggiungiCit} (POST): indirizzo usato per aggiungere una città dai preferiti;
    \item \emph{/rimuoviCit} (PUT): indirizzo usato per rimuovere una città dai preferiti;
    \item \emph{/logout} (GET): indirizzo usato per eliminare il cookie usato per autenticare l'utente;
    \item \emph{/aree} (GET): pagina per mostrare le previsioni delle diverse aree geografiche (continenti);
    \item \emph{/aree/area} (GET): indirizzo raggiunto per ottenere i paesi e le capitali di una data area.
\end{itemize}

\subsection{Descrizione delle risorse}

\subsubsection{Database}

Il motore utilizzato per realizzare il database è MongoDB.\\
Il database si compone di un'unica collezione (chiamata \emph{users}), utilizzata per salvare i dati relativi agli utenti registrati.
Ogni documento, come anticipato nell'introduzione, avrà la seguente struttura:
\begin{itemize}
    \item \emph{\_id}: id univoco assegnato dal database in automatico;
    \item \emph{user}: username;
    \item \emph{email}: email dell'utente;
    \item \emph{pwd}: password dell'utente sotto forma di digest SHA256;
    \item \emph{pref}: array contenente le città messe tra i preferiti dall'utente. può essere vuoto.
\end{itemize}

\subsubsection{Autenticazione}

L'autenticazione sarà considerata corretta se il digest SHA256 della password e l'email inseriti dall'utente corrisponderanno
a quelli di un documento presente nel database (si ricorda che username e email sono \textbf{univoci}).

\vspace{5mm}

Nel caso l'autenticazione vada a buon fine, verrà resituito un \emph{cookie} valido per 30 minuti e contente lo username dell'utente.
Se l'utente dovesse cambiare il suo username all'interno dell'area personale, il contenuto del cookie verà aggiornato.
\chapter{Codice}

\section{HTML 5}

Il codice di tutte le pagine dell'applicazione è strutturato in modo da avere:
\begin{itemize}
    \item lo stesso contenuto nel tag head;
    \item una navbar contenente:
          \begin{enumerate}
              \item il nome del sito (Meteo) e la sua icona;
              \item un link alla Home in modo da poterci sempre tornare in qualunque momento;
              \item un menù a tendina per il meteo in tempo reale delle diverse aree geografiche (continenti);
              \item un menù a tendina per autenticarsi/registrarsi oppure per accedere all'area personale/effettuare
                    il logout nel caso in cui si sia già autenticati;
              \item un toggle per scegliere tra light e dark mode;
              \item la barra di ricerca per ottenere il meteo in tempo reale di una città.
          \end{enumerate}
          I punti 3 e 6 non saranno presenti nella navbar delle pagine dedicate a login e registrazione.
\end{itemize}

\vspace{5mm}

Ogni pagina avrà poi un contenuto personalizzato in base al suo scopo.

\section{CSS3}

Nel progetto, oltre ad utilizzare le classi di \emph{Bootstrap} per gestire gli stili, sono stati integrati due file \emph{css} che si possono trovare alla repository github
\url{https://github.com/matildelabruzzo/ProgettoPwm} seguendo il percorso \emph{assets/css}.

\section{API}

Nel progetto sono state integrate 4 API:
\begin{itemize}
    \item \emph{openWeather}: API usata per ottenere le condizioni meteo in tempo reale;
    \item \emph{pexels}: API utilizzata per reperire le immagini relative alle città mostrate nelle pagine a partire da una query contenente un filtro che descrive la città mostrata;
    \item \emph{spott}: API utilizzata per ottenere i nomi delle città mondiali e per permettere la ricerca incrementale;
    \item \emph{rest countries}: API utilizzata per ottenere i nomi delle capitali delle diverse aree geografiche.
\end{itemize}

\section{Node.js}

Il server Node.js, implementato secondo paradigma \emph{RESTful}, mette in pratica tutte le funzionalità descritte nei capitoli precedenti. Viene implementato
grazie all'uso del framework \emph{Express} e di \emph{MongoDB} per memorizzare i dati degli utenti registrati.\\
La comunicazione avviene tramite l'uso di HTTP e le chiamate asincrone; le chiamate effettuate dal server al database sono anch'esse tutte asincrone.

\vspace{5mm}

Si può utilizzare un file presente in \emph{assets/mongoDB} per configurare il database in automatico.

\vspace{5mm}

Per consultare il file server.js contenente il server in toto e il relatvo codice,
riferirsi alla repository github \url{https://github.com/matildelabruzzo/ProgettoPwm} seguendo il percorso \emph{assets/node}.

\section{Librerie esterne e JavaScript}

Per consultare tutti i file JavaScript e il relatvo codice, riferirsi alla repository github \url{https://github.com/matildelabruzzo/ProgettoPwm}
nel percorso \emph{assets/js}.

L'unica libreria esterna integrata è utilizzata per il calcolo dello SHA256 di una stringa è reperibile al link \url{https://cdnjs.cloudflare.com/ajax/libs/crypto-js/3.1.2/rollups/sha256.js}.
\chapter{Possibili sviluppi e conclusioni}

L'applicazione si occupa di fornire le condizioni meteo in tempo reale relative ad una o più città in tutto il mondo.

\vspace{5mm}

Alcuni possibili spunti di sviluppo per accrescere l'efficienza, la sicurezza e l'estetica dell'applicazione potrebbero essere:
\begin{itemize}
    \item introduzione di librerie più efficienti rispetto a quelle attualmente integrate;
    \item efficentamento del codice JavaScript attualmente integrato attraverso un maggiore studio delle funzioni offerte ed eventuale
          introduzione di apposite librerie per svolgere tali scopi;
    \item introduzione di web worker per effettuare le richieste asincrone al server tramite fetch, caricare le immagini e effettuare
          il calcolo dello SHA256 delle password inserite dall'utente;
    \item introduzione di un sistema di comunicazione con chiavi SSL per l'utilizzo del protocollo HTTPS;
    \item introduzione di una navbar apposita per i dispositivi mobili.
\end{itemize}

\vspace{5mm}

Alcuni spunti di sviluppo per accrescere il valore commerciale dell'applicazione sono riassumibili in:
\begin{itemize}
    \item introduzione di pubblicità all'interno del sito;
    \item introduzione di un'API per fornire ad altri sistemi il meteo in tempo reale.
\end{itemize}
\chapter{Bibliografia}

\begin{itemize}
    \item API pexels: \url{https://www.pexels.com};
    \item API openWeather: \url{https://www.pexels.com};
    \item API città: \url{https://www.spott.dev/};
    \item API paesi e capitali: \url{https://restcountries.com/};
    \item Libreria usata per generare digest SHA256: \url{https://cdnjs.cloudflare.com/ajax/libs/crypto-js/3.1.2/rollups/sha256.js};
    \item Repository github con l'intero progetto e la documentazione con LaTex sorgente: \url{https://github.com/matildelabruzzo/ProgettoPwm};
    \item Bootstrap: \url{https://getbootstrap.com};
    \item Swiper API: \url{https://swiperjs.com/swiper-api};
\end{itemize}

\end{document}