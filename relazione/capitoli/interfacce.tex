\chapter{Interfacce}

Per realizzare le pagine sono stati usati \emph{HTML 5} e \emph{bootstrap}.\\
Gli elementi di personalizzazione, come il nome utente nella navbar una volta autenticati o il meteo
in tempo reale delle città salvate tra i preferiti, sono inseriti utilizzando \emph{EJS}.

\section{Index.ejs}

Nella parte superiore dell'index come prima cosa troviamo una navbar semplice con elementi completamente intuitivi: a partire da sinistra si ha il nome e il logo dell'applicazione,
e il tasto home; grazie al menù a tendina \emph{Aree} è possibile navigare per aree geografiche (i continenti) e cercare informazioni sulle città relative ad esse; cliccando invece sul menù {Autenticazione}
l'utente può accedervi o registrarsi; a seguito del login, questo menù viene sostituito con un altro per poter effettuare il logout e per poter accedere all'area personale, dove sono salvate le città
preferite dall'utente e dove quest'ultimo può cambiare le informazioni di accesso. \\
Sempre sulla navbar troviamo un toggle che dà la possibilità all'utente di cambiare la modalità di visualizzazione,
\emph{light mode} o \emph{dark mode} a seconda della sua preferenza. Sulla destra troviamo invece una barra di ricerca.

\vspace{5mm}

Al di sotto della navbar troviamo